\begin{table}[H]
    \begin{center}
    \caption{Descrição das propriedades envolvidas no experimento}
    \label{tab:Descrição das propriedades do grafeno}
    \resizebox{16cm}{!}{
    \begin{tabular}{ l | l }
    \toprule
	\textbf{Propriedade} & \textbf{Descrição} \\ 
	\midrule
    \bm{$n_x$} & número de átomos no eixo X, ou seja, é o comprimento da nanofita \\
    \bm{$n_y$} & número de átomos no eixo Y, ou seja, é a largura da nanofita  \\
    \bm{$g_x$} & número de deformações gaussianas ao longo do eixo X na nanofita de grafno \\
    \bm{$d_1$} & distância os centros de deformações gaussianas vizinhas na nanofita de grafeno \\
    \textbf{b} & largura das deformações gaussiana \\
    \textbf{alpha} & relação entre a altura e a largura da deformações gaussianas, ou seja, $\alpha = (A/b)^2$. Está relacionado com o \textit{strain} produzido por uma deformação gaussiana \\
    \textbf{Energy} & energia dos elétrons incidentes \\
    \textbf{Conductance} & condutância elétrica $G(2e^2/h)$ \\
    \textbf{Transmission} & probabilidade de transmissão por vale \\
    \bottomrule % <-- Bottomrule here
    \end{tabular}
    }
    \end{center}
\end{table}