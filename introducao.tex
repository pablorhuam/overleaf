\chapter{INTRODUÇÃO}
\label{sec:introducao}
O uso de deformações mecânicas para melhorar as propriedades eletrônicas e ópticas dos materiais é uma área relativamente nova. Deformações mecânicas são usadas para enfraquecer a dispersão intervale, reduzir a massa efetiva dos buracos nos lasers semicondutores III-IV e aumentar a mobilidade eletrônica nos transístores. Desde 2002 a Intel incorpora diversas técnicas para criar uma deformação de aproximadamente 0,1\% em transístores com comprimento de canal menor que 90 nm. Este cenário foi substancialmente enriquecido com o advento do grafeno, uma membrana de espessura atômica que apresenta propriedades eletrônicas, ópticas e mecânicas surpreendentes. 
Do ponto de vista mecânico, o grafeno tem o módulo tensional mais alto ($\approx$ 1 terapascal) e taxas de deformações elásticas de aproximadamente 20\% são facilmente alcançadas; além disso, dispositivos padronizados podem atingir deformações elásticas ainda maiores ($\approx$ 200\%). 

Do ponto de vista eletrônico, os elétrons no grafeno sentem as deformações da rede através de um acoplamento com um vetor potencial pseudomagnético adicional. A conseqüência direta deste acoplamento é que as deformações locais não-uniformes, como deformações Gaussianas, se traduzem diretamente em um campo pseudomagnético efetivo; a magnitude deste campo pseudomagnético pode facilmente atingir mais de 300 Tesla, atestando o forte impacto que as deformações Gaussianas têm nas propriedades eletrônicas.

Nos últimos anos, as metodologias de machine learning (ML) surgiram como uma nova ferramenta na física e na ciência dos materiais. As aplicações incluem previsão da estrutura atômica e predição de propriedades físicas.

Experimentos em grafeno sobre nanopilares ou nanoesferas têm mostrado ser a forma mais efetiva de gerar deformações gaussianas em nanofitas de grafeno; a superrede gerada é de milhares/milhões de átomos, dificultando assim o estudo teórico necessário para entender e predizer as propriedades de transporte eletrônico nesses sistemas. Do ponto de vista teórico, as funções de Green (FG) são as ferramentas padrão para estudar as propriedades de transporte eletrônico em sistema de baixa dimensionailidade. As FG conseguem resolver problemas de transporte balístico, incluindo deformações mecânicas, diferentes tipos de desordem e efeitos de muitos corpos; também pode-se calcular propriedades locais tais como a densidade local de estados e densidade de corrente. Contudo, o tamanho dos sistemas a serem estudados é limitado a alguns milhares de átomos, dificultando assim as predições teóricas sobre as superedes de deformações Gaussianas.

Inúmeros avanços foram e são feitos no desenvolvimento de sistemas inteligentes. Pesquisadores de diversas áreas estão usando redes neurais artificiais (RNA) para resolver uma variedade de problemas em reconhecimento de padrões, predição, otimização entre outros.  As redes neurais artificiais foram inspiradas na estrutura e no funcionamento dos neurônios biológicos. Cada nó (neurônio) é responsável por aplicar uma função de ativação em uma ou mais entradas e retornar uma ou mais saídas. As interconexões das camadas (conjunto de nós no mesmo nível) são os pesos (sinapses) e estes são ajustados iterativamente de acordo com a saída da rede esperada por meio do algoritmo de retro-propagação. Ou seja, a rede neural é capaz de ajustar os seus pesos (aprender) dado um conjunto de dados de treinamento. Diferentemente dos sistemas especialistas, as redes neurais não dependem de algoritmos específicos. Além disso, seu grande potencial está em inferir a(s) saída(s) (classe(s) ou valor(es) numérico(s)) de um exemplar nunca antes visto.

O presente trabalho visa treinar uma rede neural que seja capaz de predizer/inferir propriedades como transporte de vale de nanofitas de grafeno com diferentes configurações em termos de largura, comprimento, número de deformações gaussianas, proporção entre a altura e largura das gaussianas e distância entre elas. Para isso, são aplicadas técnicas de análise e pré processamento de dados como validação cruzada e medição de performance.