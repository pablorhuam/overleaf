% \bibliography{references_abstract.bib}

% resumo em português
\setlength{\absparsep}{18pt} % ajusta o espaçamento dos parágrafos do resumo
% \begin{resumo}
% % Vários estudos independentes mostraram que as propriedades eletrônicas do grafeno, como a condutância e transmissão por vale, variam de acordo com as características físicas da fita de grafeno. Ou seja, fitas mais/menos largas, mais/menos compridas e com mais ou menos deformações gaussianas resultam em diferentes valores de condutância e transmissão por vale. O presente trabalho apresenta a modelagem de rede neural artificial capaz de simular e predizer a correlação destas e outras configurações e seus efeitos nas propriedades eletrônicas. Conclui-se que é conveniente e fácil de usar redes neurais artificiais em experimentos numéricos para revisar os efeitos das diferentes configurações físicas de nanofitas de grafeno em relação a sua condutância elétrica e transmissão por vale.

%  \textbf{Palavras-chave}: grafeno. mlp. redes neurais artificiais. transporte eletrônico
% %  Segundo a \citeonline[3.1-3.2]{NBR6028:2003}, o resumo deve ressaltar o
% %  objetivo, o método, os resultados e as conclusões do documento. A ordem e a extensão
% %  destes itens dependem do tipo de resumo (informativo ou indicativo) e do
% %  tratamento que cada item recebe no documento original. O resumo deve ser
% %  precedido da referência do documento, com exceção do resumo inserido no
% %  próprio documento. (\ldots) As palavras-chave devem figurar logo abaixo do
% %  resumo, antecedidas da expressão Palavras-chave:, separadas entre si por
% %  ponto e finalizadas também por ponto.

% %  \textbf{Palavras-chave}: latex. abntex. editoração de texto.
% \end{resumo}

% resumo em inglês
\begin{resumo}[Abstract]
The control of the valley degree of freedom of electrons (valleytronics)  has recently emerged as a promising technology for the next generation of electronic devices; this quantum  number naturally appears in periodic solids with  degenerated local minima and maxima at inequivalent  points of the Brillouin zone. Similar to spintronics, the applicability of valleytronics relies on the electric generation, control and detection of valley currents \citeauthor{Valley2D}; %in this context, 2D materials with hexagonal lattices such as graphene and transition metal dichalcogenides offer two valleys (K and K’) well separated in momentum space that can be accessed by optical \cite{PhysRevB.77.235406,Cao:2012fk}, magnetic \cite{PhysRevLett.113.266804, PhysRevLett.114.037401} and mechanical means \cite{nl400872q,7b01663}. For example, it has been observed that under broken spatial inversion symmetry these systems generate valley-dependent optical selection rules \cite{Lee:2016kx} and dissipation less topological valley current \cite{Gorbachev448}. However, despite the success achieved, this approach is limited to high quality samples with perfect alignment of the layers. On the other hand, electrons in opposite valleys in graphene see in homogeneous mechanical deformation as regions with opposite polarity pseudomagnetic fields; pseudo magnetic fields exceeding 300 T have been observed in out-of-plane deformations \cite{Levy544}, numerical studies of Gaussian bubbles \cite{PhysRevLett.117.276801} have shown separation of valley currents and valley filtering, unfortunately, the observed effects require fine tuning of the energy, defined height/width ratio of the bubble, narrow contacts, location of the nanobubble near to the right contact and crystalline orientation.  All these ingredients mix in a subtle and hidden way that it is impossible to predict the effect of a given deformation before heavy calculations.
\end{resumo}

